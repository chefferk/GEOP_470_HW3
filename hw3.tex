\documentclass[titlepage, 11pt]{article}
\usepackage{amsmath,amssymb,amsthm,latexsym,paralist}
\usepackage[margin=1in]{geometry}
\usepackage{csquotes}

\theoremstyle{definition}
\newtheorem{problem}{Problem}
\newtheorem*{solution}{Solution}

\author{Jonathan Foster \\ Keaton Cheffer \\ Kristin Moczygemba}
\title{Geophysics 470 \\ Homework 3: Heat in a Spherical Body}
\date{2/14/18}

\begin{document}

\maketitle

% Problem 1
\begin{problem}\ \\
T\&S Problem 4-14: What would the central temperature of the Earth be if it were modeled by a sphere  with uniform volumetric heating?\vspace{\baselineskip}

\noindent
You don't have to start from scratch, use the solution for $T(r)$ with a fixed surface temperature that is given to you (Eq. 4.42).
\noindent
Note that the problem doesn't give you $\rho H$, you will have to express the internal heating rate in terms of the surface heat flow (the discussion in section 4.9 covers this).
\end{problem}

\begin{solution}\ \\

\begin{tabular}{ll}
Let: & $q_{0}=70$    mWm$^{-2}$\\
     & $k=4$         Wm$^{-1}$K$^{-1}$\\
     & $T_{0}=300$   K\\ 
     & $a = 6370800$ m\\ 
\end{tabular}\vspace{\baselineskip}

Surface heat flux at $r = a$ is:

$${q}_{0}\left[ \frac{mW}{m^2} \right] = \frac{1}{3} \rho \left[\frac{kg}{{m}^{3}}\right] H\left[\frac{mW}{kg}\right] a\left[m\right]$$ 

Solving for $\rho H$, we get:

$$\rho H \left[ \frac{mW}{m^3} \right] = \frac{3{q}_{0}}{a}\left[\frac{mW}{m^3} \right]$$

Central temperature of a sphere with $r = a$ is:
\begin{align*}
T\left[K \right] & = {T}_{0}\left[ K \right] + \frac{\rho H\left[\frac{mW}{m^3} \right]}{6k\left[ \frac{W}{m K} \right]} {a}^{2}\left[m^2 \right]\\
& = {T}_{0}\left[ K \right] + \frac{\rho H}{6k} {a}^{2} \left[\frac{mWK}{W} \right]\\
& = {T}_{0}\left[ K \right] + \frac{\rho H}{6k} {a}^{2} \left[\frac{K}{1000} \right]
\end{align*}

Substituting in $\rho H$, we get:

\begin{align*}
T & = {T}_{0} + \frac{{q}_{0}}{2(1000k)a} {a}^{2}\\
& = {T}_{0} + \frac{{q}_{0}a}{2000k}
\end{align*}

Plugging in values, we get:

\begin{align*}
T & = 300 + \frac{70\times6370800}{2000\times4} \\
T & = \boxed{56044.5 \ K}
\end{align*}

\end{solution}

\newpage

% Problem 2
\begin{problem}\ \\
(a) Suppose that a spherical planet is broken into two layers, core and mantle, each of which has a different heat production rate, $\rho H$ and a different conductivity, $k$. Calculate the temperature profile in this planet, where the core radius is $b$, and the planet radius is $a$. The surface temperature is fixed at Ts. This is a spherical version of the two-layer problem you did in HW2; now the bottom layer is a sphere and the top layer is a spherical shell. In each layer, the general spherical solution (Eq. 4.40) applies, so don’t derive it!

$$T = - \frac{\rho H}{6k}{r}^{2} + \frac{c_1}{r} + c_{2} \qquad \qquad (4.40)$$

\noindent
You just have to write down the boundary conditions to find expressions for the integration constants.\vspace{\baselineskip}

\noindent
There’s a lot of algebra here, but you will use this solution in the next 3 problems.\vspace{\baselineskip}

\noindent
(b) Write an expression for $T(0)$, the temperature at the center of the planet. It will be advantageous for you to write your solution with terms that have the ratio of $k$ values. To do this, you multiply one of the $k$’s (I use the core conductivity, $k_c$) by 1, but a particular version of 1:

$$k_{c} = k_{c} \left(\frac{k_{m}}{k_{m}}\right) = k_{m}\left(\frac{k_{c}}{k_{m}}\right)$$

\noindent
So if I make this substitution everywhere, then gather together terms and that have common factors of $k_{m}$, its easier to see what the role of two different conductivities is. 
\end{problem}

\begin{solution}\ \\
\noindent
(a) 
Steady-State Conservation:

$$0 = k_c \left(\frac{d^2 T_c}{dr^2} + \frac{2}{r} \frac{dT_c}{dr}\right) + \rho H_c$$

$$0 = k_m \left(\frac{d^2 T_m}{dr^2} + \frac{2}{r} \frac{dT_m}{dr}\right) + \rho H_m$$

General Solutions:

$$T_c = -\frac{\rho_c H_c}{6k_c}r^2 + \frac{c_1}{r} + c_2$$

$$T_m = -\frac{\rho_m H_m}{6k_m}r^2 + \frac{c_3}{r} + c_4$$

Temperature Gradients:

\begin{align*}
    q_c & = -k_c \frac{dT_c}{dr}\\
    & = \frac{\rho_c H_c}{3}r + \frac{c_1 k_c}{r^2}
\end{align*}

\begin{align*}
    q_m & = -k_m \frac{dT_m}{dr}\\
    & = \frac{\rho_c H_m}{3}r + \frac{c_3 k_m}{r^2}
\end{align*}

\newpage
\noindent
Boundary Conditions:
\begin{enumerate}
    \item Same temperature at layer interface: $T_m (r=b) = T_c (r=b)$
    \item Same heat flux at layer interface: $q_m (r=b) = q_c (r=b)$
    \item Temperature gradient approaches zero as $r$ approaches zero: $\frac{dT_c}{dr}(r=0)=0 \therefore \boxed{c_1 =0}$
    \item Fixed temperature at the surface: $T_m (r=a) = T_s$
\end{enumerate}

\noindent
Applying Boundary Conditions:\\

Using the general solution for $T_c$ and boundary condition (3), we get:

$$T_c = -\frac{\rho_c H_c}{6k_c}r^2 + c_2$$

At boundary condition (1) we get:

$$T_c (b) = -\frac{\rho_c H_c}{6k_c}b^2 + c_2$$

Solving for $c_2$ we get:

$$ c_2 = T_b + \frac{\rho_c H_c}{6k_c}b^2 $$

Plugging $c_2$ back into $T_c$, we get:

$$\boxed{ T_c = \frac{\rho_c H_c}{6k_c}\left( b^2 - r^2 \right) + T_b}$$

Using the boundary condition (2), we see that:

$$-\frac{k_c dT_c}{dr} = -\frac{k_m dT_m}{dr}$$

Using temperature gradients we get:

$$-\frac{\rho_c H_c}{3}r = -\frac{\rho_m H_m}{3}r - \frac{k_m c_3}{r^2}$$

Solving for $c_3$, we get:

$$c_3 = \frac{\rho_c H_c}{3k_m r}b^3 - \frac{\rho_m H_m r}{3k_m}b^3$$

Plugging $c_3$ back into the general solution for $T_m$ we get:

$$T_m = -\frac{\rho_m H_m}{6k_m}r^2 + \frac{\rho_c H_c}{3k_m r}b^3 - \frac{\rho_m H_m}{3k_m r}b^3 + c_4$$

At boundary condition (4) we get:

$$T_m (a) = -\frac{\rho_m H_m}{2k_m}a^2 + \frac{\rho_c H_c}{3k_m}a^2 + c_4$$

\newpage
Solving for $c_4$ we find:

$$c_4 = T_s + \frac{\rho_m H_m}{2k_m}a^2 - \frac{\rho_c H_c}{3k_c}a^2$$

Finally plugging $c_4$ back into $T_m$, we get:

$$\boxed{ T_m = T_s - \frac{\rho_m H_m}{6k_m}r^2 + \frac{\rho_c H_c}{3k_m r}b^3 - \frac{\rho_m H_m}{3k_m r}b^3 - \frac{\rho_c H_c}{3k_m}a^2 + \frac{\rho_m H_m}{2k_m}a^2}$$

\noindent
(b)\\

$k_c = k_c\left(\frac{k_m}{k_m}\right) = k_m\left( \frac{k_c}{k_m} \right)$\\

Solving for $T(b)$ we get:
\begin{align*}
T_b & = T_s -\frac{\rho_m H_m}{6k_m}(b)^2 + \frac{\rho_c H_c}{3k_m (b)}b^3 - \frac{\rho_m H_m}{3k_m (b)}b^3 - \frac{\rho_c H_c}{3k_m}a^2 + \frac{\rho_m H_m}{2k_m}a^2\\
& = T_s + \frac{\rho_m H_m}{2k_m}\left( a^2 - b^2 \right) + \frac{\rho_c H_c}{3k_m}\left(b^2-a^2\right)
\end{align*}

Plugging $T_b$ into our original $T_c(r)$ equation we get:


    $$T(r) = T_s + \frac{\rho_c H_c}{6k_c}\left( b^2 - r^2 \right) + \frac{\rho_m H_m}{2k_m}\left( a^2 - b^2 \right) + \frac{\rho_c H_c}{3k_m}\left( b^2 - a^2 \right)$$
    
Finally plugging 0 in we get:

    $$\boxed{T(0) = T_s + \frac{\rho_c H_c}{6k_c}b^2 + \frac{\rho_m H_m}{2k_m}\left( a^2 - b^2 \right) + \frac{\rho_c H_c}{3k_m}\left( b^2 - a^2 \right)}$$


\end{solution}

\newpage

% Problem 3
\begin{problem}\ \\
T\&S 4-16: It is assumed that a constant density planetary body of radius $a$ has a core of radius $b$. There is uniform heat production in the core but no heat production outside the core. Determine the temperature at the center of the body in terms of $a, b, k, T_0$ (the surface temperature), and $q_0$ (the surface heat flow).\vspace{\baselineskip}

\noindent
This is just a special case of the Problem 2, with the $k$ values equal and one of the $\rho H = 0$. You can easily modify your solution from 2 to write down $T(r)$ so don't set up and solve the problem all over again! Make a sketch of the solution, $T(r)$ and write the expression for $T(r=0)$.
\end{problem}

\begin{solution}\ \\

\noindent
$k_c = k_m$\\
$\rho_1 = \rho_2$\\
$H_m = 0 \to \rho H_m = 0$\\

At $H_m = 0$:

$$T_c = \frac{\rho H_c}{6k_c}\left( b^2 - r^2 \right) + \frac{\rho b^2}{3k_m}(H_c)(1-\frac{b}{a}) + T_s$$

$$T_m = \frac{\rho b^3}{3k_m}H_c \left( \frac{1}{r} - \frac{1}{a} \right) + T_s$$

At $r=0$:

$$T_c(r=0) = \frac{\rho H_c}{6k_c}b^2 +\frac{\rho H_c}{3k_m}b^2 - \frac{\rho H_c }{3ak_m}b^3 + T_s$$

At  $k_c = k_m$:

\begin{align*}
    T_c(r=0) & = \frac{\rho H_c}{6k}b^2 + \frac{\rho H_c}{3k_m}b^2 - \frac{\rho H_c}{3ak}b^3\\
    & = \frac{\rho H_c}{2k}b^2 - \frac{\rho H_c}{3ak}b^3 + T_s
\end{align*}

At $q_0 = \frac{1}{3}\rho H a$:
\begin{align*}
    T_c(r=0) & = \frac{3q_0}{2ak}b^2 - \frac{2q_0}{3 a^2 k}b^3 + T_s\\
    & = \frac{3q_0}{2ak}b^2 - \frac{q_0}{a^2k}b^3 + T_s
\end{align*}

$T_s = T_0$:

$$\boxed{T_c(r=0) = \frac{q_0 b^3}{a^2k}\left( \frac{3a}{2b} - 1 \right) + T_0}$$

\end{solution}

\newpage

% Problem 4
\begin{problem}\ \\
T\&S Problem 4-17: Determine the steady-state conduction temperature profile for a spherical model of the Moon in which all the radioactivity is confined to an outer shell whose radii are $b$ and $a$ ($a$ is the lunar radius). In the outer shell $H$ us uniform.\vspace{\baselineskip}

\noindent
This is another special case of the Problem 2, with equal $k$'s and one of the $\rho H = 0$. Make a sketch of the solution, $T(r)$ and write the expression for $T(r = 0)$. Using the numbers form the text section 4.10 ($T_0 = 250$ K, $q_0 = 18\times{10}^{-3}$ W/m$^{2}$, lunar radius = 1738 km) and assuming the layer contain the radioactivity is 100 km thick, calculate first $\rho H$ in the outer layer and then temperature at the center of the moon. Compare this to the temperature calculated in the text for a uniform Moon and explain in words why one case is cooler than the other.
\end{problem}

\begin{solution}\ \\

\begin{tabular}{ll}
Let: & $q_{0}=18\times 10^{-3}$     mWm$^{-2}$\\
     & $k=3.3$                    Wm$^{-1}$K$^{-1}$\\
     & $T_{0}=250$                K\\ 
     & $a = 1738000$              m\\
     & $b = 1638000$              m
\end{tabular}\vspace{\baselineskip}

$$T(0) = \frac{\rho_m H_m}{2k}(a^2 - b^2) +T_0$$

$$q_0 = \frac{\rho H}{3}(a-b)$$

Solving for $\rho H$ we get:

$$\rho H = \frac{3q_0}{a-b}$$

Plugging our boundary conditions into $\rho H$ we get:

\begin{align*}
    \rho H & = \frac{3(18 \times 10^{-3})}{(1738000 - 1638000)}\\
    & = 5.4 \times 10^{-7}
\end{align*}

Plugging the remaining boundary conditions into our original equation gives us:

\begin{align*}
    T(0) & = \frac{5.4\times 10^{-7}}{2(3.3)}\left( 1738000^2 - 1638000^2 \right) + 250\\
    & = \boxed{27872 K}
\end{align*}

\end{solution}

\end{document}
