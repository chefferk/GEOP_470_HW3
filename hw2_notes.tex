\documentclass[12pt]{article}
\usepackage{amsmath,amssymb,amsthm,latexsym,paralist}
\usepackage[margin=1in]{geometry}
\usepackage{csquotes}

\title{Homework 3: Notes}
\setlength\parindent{0pt}

\begin{document}

\maketitle

The effects of spherical geometry are not so important for the Earth’s lithosphere since it is very thin compared to the radius\\

In a small body like the moon, the lithosphere can be a substantial fraction of the radius\\

The total heat flow out of the shell through its outer surface is:

$$4\pi {\left( r+\delta r \right)}^ { 2} {q}_{r} \left( r + \delta r \right)$$

The total heat flow into the shell at its inner surface is:

$$4 \pi {r}^{2} {q}_{r} \left(r \right)$$

$q = $ heat flux\\
$q_{r} = $ radial head flux\\

Since $\delta_{r}$ is infinitesimal, we can expand $q_{r} \left(r + \delta r \right)$ in a Taylor series as:

$${ q }_{ r }\left( r+\delta  \right) ={ q }_{ r }\left( r \right) +\delta r\frac { dq_{ r } }{ dr } +... \qquad \qquad \left ( 4.34 \right)$$

Neglecting powers of $\delta_{r}$, the net heat flow out of the spherical shell is given by:

$$4 \pi \left[ {\left( r + \delta r \right)}^{2} {q}_{r} \left( r + \delta r \right) - {r}^{2} {q}_{r} \left( r \right) \right] $$
$$= 4 \pi {r}^{2} \left(\frac{2}{r} {q}_{r} + \frac{d {q}_{r}}{dr}\right) \delta_{r} \qquad \qquad (4.35)$$

$H = $ rate of heat production per unit mass [W/kg]\\

Total rate at which heat is produced in the spherical shell is:

$$4 \pi {r}^{2} \rho H \delta r$$

$4 \pi {r}^{2} \delta r = $ the approximate expression for the volume of the shell\\

If we set the rate of heat production equal to the net heat flow out of the spherical shell, we get (heat balance equation):

$$4 \pi {r}^{2} \rho H \delta r = 4 \pi {r}^{2} \left(\frac{2}{r} {q}_{r} + \frac{d {q}_{r}}{dr}\right) \delta_{r}$$
$$\frac{d{q}_{r}}{dr} + \frac{2 {q}_{r}}{r} = \rho H \qquad \qquad (4.36)$$

Fourier's law in spherical geometry:

$${q}_{r} = -k \frac{dT}{dr} \qquad \qquad (4.37)$$

Applying Fourier's law $(4.37)$ to the heat balance equation $(4.36)$, we get:

$$0 = k \left(\frac{{d}^{2}T}{d{r}^{2}} + \frac{2}{r} \frac{dT}{dr} \right) + \rho H \qquad \qquad (4.38)$$
or
$$0 = k \frac{1}{{r}^{2}} \frac{d}{dr} \left({r}^{2}\frac{dT}{dr} \right) + \rho H \qquad \qquad (4.39)$$

By integrating equation $(4.39)$ twice, we get a general expression for the temperature in a sphere or spherical shell with internal heat production and in steady state:

$$T = - \frac{\rho H}{6k} {r}^{2} + \frac{{c}_{1}}{r} + {c}_{2} \qquad \qquad (4.40) $$

${c}_{1}$ and ${c}_{2}$ depend on boundary conditions\\

Example:\\
$a = $ radius\\
${T}_{0} = $ outer surface temperature\\
${c}_{1} = 0$ (in order to have a finite temperature at the center of the sphere)

$${c}_{2} = {T}_{0} + \frac{\rho H {a}^{2}}{6k} \qquad \qquad (4.41)$$

Then, the temperature in the sphere is given by:

$$T = {T}_{0} + \frac{\rho H}{6k} \left( {a}^{2} - {r}^{2} \right) \qquad \qquad (4.42)$$

From equation $(4.37)$ the surface heat flux ${q}_{0}$ at $r = a$ is (conservation of energy that apples no matter what the mode of internal heat transfer in the sphere is.):

$${q}_{0} = \frac{1}{3} \rho H a \qquad \qquad (4.43)$$




\end{document}
