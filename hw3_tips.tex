\documentclass[12pt]{article}
\usepackage{amsmath,amssymb,amsthm,latexsym,paralist}
\usepackage[margin=1in]{geometry}
\usepackage{csquotes}

\title{Homework 3: Tips}

\begin{document}

\maketitle

Things to keep in mind when doing Homework 3:

\begin{compactenum}
\item We are only going to do one basic spherical problem - spherical symmetry (variation in $r$ only) with conduction and possibly heat production. Therefore, we have already solved this problem, with the general solution given in the book as Eq. 4.40. All you need to do is find the proper boundary conditions to solve for the constants.
\item If the region you are solving for includes the center of the sphere ($r = 0$), then one of the conditions MUST be that $\frac{dT}{dr} = 0$ at $r = 0$.
\item If multiple layers, the same rules apply as in Cartesian coordinate problems: (a) the equation that represents the temperature solution in one layer is different from the equation for another layer; use different notations to avoid confusion (b) at an interface, $T$ and heat flux $q$ are unknown, but they must be continuous; so the equations for the two layers will be equal at the interface value of r.
\end{compactenum}


\end{document}
